\pagenumbering{arabic}
\chapter{Introduction}

La valorisation d'une base de données passe par la développement d'une application serveur permettant de communiquer avec celle-ci mais également d'une application client effectuant des requêtes auprès du serveur. Cette application client peut souvent être une application web basée sur du JavaScript et des XML Http Requests pour la communication avec le serveur. Cependant, même si ce type d'application présente beaucoup d'avantages, les normes du web et ses implémentations peuvent présenter des subtilités d'un navigateur à l'autre. Ce processus peut rendre le développement coûteux en temps, surtout lorsque l'on recherche des applications performantes. De plus, la phase de test nécessite d'être d'autant plus minutieuse pour vérifier la compatibilité avec tous les navigateurs. Toutes ces contraintes peuvent pousser à utiliser des kits de développement tels que GWT (\textit{Google Web Toolkit}) qui permettent de faciliter le développement de l'application et son portage sur tous le navigateurs.

Dans ce projet, nous allons donc réaliser la valorisation d'une base de données en utilisant \textit{Google Web Toolkit} et \textit{Google App Engine}. L'application réalisée pourra donc être déployée dans le Cloud de Google.

Dans ce rapport, nous aborderons dans un premier temps le serveur développé avec GWT, ensuite nous verrons les différents aspects du client et enfin nous étudierons plus en détails l'application réalisée.